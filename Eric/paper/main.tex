\documentclass{article}
\usepackage{graphicx} % Required for inserting images
\usepackage{authblk} % Required for advanced author formatting
\usepackage{glossaries} % for abbreviations and glossary
\usepackage[backend=biber,style=chicago-authordate]{biblatex} % chicago style

\makeglossaries % needed for any glossaries
% when using a glossary word or abbreviation use the following
% \glos{word}
% This ensures it is cited properly in the glossary

\title{CSCD350 - Final Project Report \\
Team 10: UwUltimate Stardust Crusaders \\
\large Dr. Sanmeet Kaur \\
Department of Computer Science and Electrical Engineering\\
June 2024}

\author[1]{Alexa Darrington - adarrington@ewu.edu}
\author[1]{Eric Leachman - eleachman@ewu.edu}
\author[1]{Will Reese - wreese@ewu.edu}
\author[1]{Lewis Thomas - lthomas32@ewu.edu}
\author[1]{Dennis Vinnikov - dvinnikov@ewu.edu}

\affil[1]{Eastern Washington University}

\renewcommand\Authsep{\quad} % Separation between authors in the same row
\renewcommand\Authands{\quad} % Separation between the last two authors in the same row
\renewcommand\Authfont{\normalsize} % Font size for author names
\renewcommand\Affilfont{\small} % Font size for affiliation

\begin{document}

\maketitle

\begin{figure}[h]
    \centering
    \includegraphics[width=.8\linewidth]{images/logo.png}
\end{figure}

%--------------------------------------------------------------------------
%--------------------------------------------------------------------------

\newpage

%--------------------------------------------------------------------------
%--------------------------------------------------------------------------

\section{Introduction}

The introduction provides a comprehensive description of the project, summarizing key elements from the Project Description document, including the project's evolution, background, related work, and an overview. It identifies the clients and stakeholders, detailing their preferences. This document aims to summarize the project's progress and outline the technical details of the engineering efforts undertaken. 

    \subsection{Project Overview}

    \subsection{Problem Definition and Scope}

    \subsection{Assumptions and Constraints}

    \subsection{Objectives}

    \subsection{Methodology Used}

    \subsection{Project Outcomes and Deliverables}

    \subsection{Novelty of Work}

%--------------------------------------------------------------------------
%--------------------------------------------------------------------------

\newpage

%--------------------------------------------------------------------------
%--------------------------------------------------------------------------

\section{Team Members}

Each team member's narrative entry highlights their name, degree plan, project role, areas of experience, and technical interests, demonstrating the team's skills and project coverage. For example, Aaron Crandall, a computer science student, focuses on developing the Gamma Module, leading user experience feedback, and has skills in C/C++, Python, and Genetic Algorithms. 

    \subsection{Alexa Darrington}

    \subsection{Eric Leachman}

    \subsection{Will Reese}

    \subsection{Lewis Thomas}

    \subsection{Dennis Vinnikov}
    

%--------------------------------------------------------------------------
%--------------------------------------------------------------------------

\newpage

%--------------------------------------------------------------------------
%--------------------------------------------------------------------------

\section{Requirement Analysis}

This section defines the project requirements, drawing from the Project Requirements document. It includes an introductory paragraph outlining what the section covers, followed by detailed, quantified requirements expected from the final design. 

    \subsection{Overall Description}
    
        \subsubsection{Product Perspective}
        
        \subsubsection{User Stories and Use Case Diagrams}
        
        \subsubsection{Product Features}
        

    \subsection{External Interface Requirements}
    
        \subsubsection{User Interfaces}
        
        \subsubsection{Hardware Interfaces}
        
        \subsubsection{Software Interfaces}
        
    
    \subsection{Other Non-functional Requirements}
    
        \subsubsection{Performance Requirements}
        
        \subsubsection{Safety Requirements}
        
        \subsubsection{Requirements}

%--------------------------------------------------------------------------
%--------------------------------------------------------------------------

\newpage

%--------------------------------------------------------------------------
%--------------------------------------------------------------------------

\section{Solution Approach}

The solution approach outlines the current design and what the project will include. It updates and integrates relevant details from the Solution Approach document, ensuring clarity and completeness in the narrative. 
    
    \subsection{Investigative Techniques}

    \subsection{Proposed Solution}

    \subsection{Tools and Technologies Used}

%--------------------------------------------------------------------------
%--------------------------------------------------------------------------

\newpage

%--------------------------------------------------------------------------
%--------------------------------------------------------------------------

\section{Prototype Design Specifications}

This new section details the prototype implementation, describing the parts and subsystems developed so far, their progress, and integration status. It includes findings from any tests performed. Visual aids such as diagrams, UI screenshots, test program screenshots, and pictures of the team working on the prototype are recommended for clarity. 

    \subsection{System Architecture}
    This subsection lists and describes the implemented functionalities, explaining any remaining work. 
    
        \subsubsection{Subsystem1}

        \subsubsection{Subsystem2}

    \subsection{Design Level Diagrams}

    \subsection{User Interface Diagrams}

%--------------------------------------------------------------------------
%--------------------------------------------------------------------------

\newpage

%--------------------------------------------------------------------------
%--------------------------------------------------------------------------

\section{Implementation and Experimental Results}

It reports test results, including screenshots if applicable, and provides CI/CD status updates. 


%--------------------------------------------------------------------------
%--------------------------------------------------------------------------

\newpage

%--------------------------------------------------------------------------
%--------------------------------------------------------------------------

\section{Conclusions and Future Directions}
This section lists major tasks for the next semester, outlining plans for their completion. 

%--------------------------------------------------------------------------
%--------------------------------------------------------------------------

\newpage

%--------------------------------------------------------------------------
%--------------------------------------------------------------------------

\listoffigures

%--------------------------------------------------------------------------
%--------------------------------------------------------------------------

\newpage

%--------------------------------------------------------------------------
%--------------------------------------------------------------------------

% glossary.tex

% Define glossary entries
\newglossaryentry{example}
{
    name=example,
    description={An example entry in the glossary}
}

\printglossary


%--------------------------------------------------------------------------
%--------------------------------------------------------------------------

\newpage

%--------------------------------------------------------------------------
%--------------------------------------------------------------------------

\printbibliography

\end{document}
